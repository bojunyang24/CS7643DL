

One important property for a feed-forward network, as discussed in the lectures, is that it must be a directed acyclic graph (DAG). Recall that a \emph{DAG is a directed graph that contains no directed cycles}. We will study some of its properties in this question.

Let's define a graph $G=(V, E)$ in which $V$ is the set of all nodes as $\{v_1, v_2, ..., v_i, ... v_n\}$ and $E$ is the set of 
edges $E = \big\{e_{i,j} = (v_i, v_j) \mid v_i, v_j \in V \big\}$.

A \emph{topological order of a directed graph} $G=(V, E)$ is an ordering of its nodes as $\{v_1, v_2, ..., v_i, ... v_n\}$ so that for every edge $(v_i, v_j)$ we have $i < j$.


There are several lemmas that can be inferred from the definition of a DAG. One lemma is: if $G$ is a DAG, then $G$ has a node with no incoming edges.   
