
%%%%%%%%%%%%%%%%%%%%%%%%%%%%%%%%%%%%%%%%%%%%%%%%%%%%%%%%%%%
\section{Multiple Choice Questions}
%%%%%%%%%%%%%%%%%%%%%%%%%%%%%%%%%%%%%%%%%%%%%%%%%%%%%%%%%%%
\begin{enumerate}[start]

\item (1 point) Consider the tables below that display infection rates for a disease in two independent regions given vaccine status.
%
\\\\
\begin{tabular}{ |c|c|c|c| } 
	\hline
	Region & Pop. & Vaccination Rates & \% of Population Infected \\
	\hline
	Cityville & 874,961 & 77.0\% & 0.36\% \\ 
	\hline
	Townsland & 578,759 & 37.7\% &  1\%\\ 
	\hline
\end{tabular}

\vspace{0.5cm}

\begin{tabular}{ |c|c|c| } 
	\hline
	Region & \% of Infected people that are Vaccinated & \% of Infected people that are Unvaccinated\\
	\hline
	Cityville & 27.8\% & 72.2\%  \\ 
	\hline
	Townsland & 5.0\% & 95.0\%\\ 
	\hline
\end{tabular}

It appears that infected individuals in Cityville are much more likely to be vaccinated than in Townsland. Given these tables, would a vaccinated individual be less likely to be infected in Cityville or Townsland?


\begin{oneparcheckboxes}
\choice Cityville
\choice Townsland
\end{oneparcheckboxes}




\item (1 point)

Given a (possibly) biased coin with $P(Heads) = p$ and $P(Tails) = 1 - p$, first determine the method to generate a fair outcome (50:50) in the fewest amount of flips using this coin.
\\\\
What is the expected number of coin flips required (in terms of $p$) to produce a fair outcome using this method?

\vspace{0.5cm}

\begin{oneparcheckboxes}
\choice $\frac{1}{p(1-p)}$
\choice $\frac{1}{1 + p^2}$
\choice $\frac{2p}{1 - p}$
\choice A fair outcome cannot be generated with a biased coin
\end{oneparcheckboxes}

\vspace{0.5cm}

\item (1 point)
$X$ is a continuous random variable with probability density function:
%
\begin{align}
p(x) = \left\{ {\begin{array}{*{20}{c}}
{\begin{array}{*{20}{c}}
{2x^3/81}&{0 \le x \le 3}
\end{array}}\\
{\begin{array}{*{20}{c}}
{2(x-3)/8}&{3\le x \le 5}
\end{array}}
\end{array}} \right.
\end{align}
%
Which of the following statements are true about the equation for the corresponding cumulative density function (CDF) $C(x)$?\\
{[\emph{Hint:} Recall that CDF is defined as $C(x) = Pr(X \le x)$.]}\\
\begin{checkboxes}
	\choice $C(x) = x^4/162$  for   $0 \le x \le 3 $
	\choice $C(x) = x^2/8 -3x/4 + 13/8$ for  $3 \le x \le 5$
	\choice All of the above
	\choice None of the above
\end{checkboxes}

\item (2 point) A random variable $x$ in standard normal distribution 
has the following probability density:
%
\begin{align}
p(x) = \frac{1}{{\sqrt {2\pi } }}{e^{ - \frac{{{x^2}}}{2}}}
\end{align}
%
Evaluate the following integral:
%
\begin{align}
\int\limits_{ - \infty }^\infty  {p(x)(a{x^2} + b{x} - c)dx}
\end{align}

[\emph{Hint:} We are not sadistic (okay, we're a little sadistic, but not for this question). This is not a calculus question.]\\

\begin{oneparcheckboxes}
	\choice a + b + c
	\choice - c
	\choice a - c
	\choice b + c
\end{oneparcheckboxes}

\vspace{2cm}

	\item
	(2 points)
	Consider the following function of $\vec{x} = (x_1, x_2, x_3, x_4, x_5, x_6)$:
	\beqn
	f(\vec{x}) = \sigma\left( \log\left( 5 \left(
	\max\{x_1, x_2\} \cdot \frac{x_3}{x_4} - (x_5 + x_6) \right)
	\right) + \frac{1}{2} \right)
	\eeqn
	where $\sigma$ is the sigmoid function
	\beqn
	\sigma(x) = \frac{1}{1 + e^{-x}}
	\eeqn
	
	Compute the gradient $\nabla_{\vec{x}} f(\cdot)$ and evaluate it at at $\hat{\vec{x}} = (-1, 3, 4, 5, -5, 7)$.
	
\begin{oneparcheckboxes}
	\choice 
	$\begin{bmatrix}
		0 \\
		0.031 \\
		0.026 \\
		-0.013 \\
		-0.062 \\
		-0.062 \\
	\end{bmatrix}$
 \choice $\begin{bmatrix}
	0 \\
	0.157 \\
	0.131 \\
	-0.065 \\
	-0.314 \\
	-0.314 \\
	\end{bmatrix}$
	\choice $\begin{bmatrix}
	0 \\
	0.357 \\
	0.268 \\
	-0.214 \\
	-0.894 \\
	-0.894 \\
	\end{bmatrix}$
	\choice $\begin{bmatrix}
	0 \\
	0.357 \\
	0.268 \\
	-0.214 \\
	-0.447 \\
	-0.447 \\
	\end{bmatrix}$
\end{oneparcheckboxes}

\item (2 points) Which of the following functions are convex?
\begin{checkboxes}
\choice $\|\x\|_{\frac{1}{2}}$ 
\choice $\min_{i=1}^k \vec{a}_{i}^T \x$ for $\x \in \mathbb{R}^n$, and a finite set of arbitrary vectors: $\{\vec{a}_1, \dots, \vec{a}_k\}$
\choice $\log{(1+\exp(\wb^T \x))} $ for $\wb \in \mathbb{R}^d$
\choice All of the above
\end{checkboxes}


\item (2 points) Suppose you want to predict an unknown value $Y \in \RF$, but you are only given
a sequence of noisy observations $x_1, \dots, x_n$ of $Y$ with i.i.d.\ noise ($x_i = Y + \epsilon_i $). If we assume the noise is I.I.D.\ Gaussian ($\epsilon_i \sim N(0, \sigma^2)$), the maximum likelihood
estimate ($\hat y$) for $Y$ can be given by: 


\begin{checkboxes}
	\choice A: $ \hat y = \argmin_y \sum_{i=1}^{n} (y - x_i)^2 $ 
	\choice B: $ \hat y = \argmin_y \sum_{i=1}^{n} |y - x_i| $
	\choice C: $\hat y = \frac{1}{n} \sum_{i=1}^{n} x_i $
	\choice Both A \&  C
	\choice Both B \& C
	
\end{checkboxes}
\end{enumerate}
\pagebreak