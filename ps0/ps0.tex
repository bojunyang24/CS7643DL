\documentclass[11pt,english, answers]{exam}

%%%%%%%%%%%%%%%%%%%%%%%%%%%%%%%%%%%%%%%%%%%%%%%%%%%%%%%%%%%
% Packages
%%%%%%%%%%%%%%%%%%%%%%%%%%%%%%%%%%%%%%%%%%%%%%%%%%%%%%%%%%%

\listfiles

\usepackage{fullpage}
\usepackage[showframe=false,margin=1in]{geometry}
\parindent=0pt
\usepackage{relsize}
\usepackage[T1]{fontenc} % for properly hyphenating words with accented chars
\usepackage[latin1]{inputenc}
\usepackage{babel}

% figure management
\usepackage{epsfig}
\usepackage{graphicx}
\usepackage{wrapfig}
\usepackage{subfig}
\usepackage[belowskip=0pt,aboveskip=0pt,font=small]{caption}
\setlength{\intextsep}{7pt plus 0pt minus 0pt}
\usepackage{tikz}

% math
\usepackage{amsmath, amsthm, amssymb}

%\usepackage{amstext}
\usepackage{textcomp}
\usepackage{stmaryrd}
\usepackage{upgreek}
\usepackage{bm}
\usepackage{cases}

% code
\usepackage{listings}

% assorted
\usepackage{url}
\usepackage{breakurl}
\usepackage[colorlinks=true]{hyperref}
\usepackage{xspace}
\usepackage{comment}
\usepackage{color}
\usepackage{afterpage}
\usepackage[normalem]{ulem}
\usepackage{enumitem}
\usepackage{xparse}


%%%%%%%%%%%%%%%%%%%%%%%%%%%%%%%%%%%%%%%%%%%%%%%%%%%%%%%%%%%
% Shortcuts
%%%%%%%%%%%%%%%%%%%%%%%%%%%%%%%%%%%%%%%%%%%%%%%%%%%%%%%%%%%
\usepackage{assets/mysymbols}
\input{assets/math_definition.tex}
\newcommand{\hide}[1]{}
\newcommand{\x}{\mathbf{x}}
\newcommand{\wb}{\mathbf{w}}
\lstset{language=C,keywordstyle={\bfseries \color{blue}}}
\NewDocumentCommand{\codeword}{v}{%
	\texttt{\textcolor{blue}{#1}}%
}

\checkedchar{\tikz\draw[black,fill=black] (0,0) circle (1.0ex);}

%%%%%%%%%%%%%%%%%%%%%%%%%%%%%%%%%%%%%%%%%%%%%%%%%%%%%%%%%%%
% Title / Author
%%%%%%%%%%%%%%%%%%%%%%%%%%%%%%%%%%%%%%%%%%%%%%%%%%%%%%%%%%%
\begin{document}

\title{CS 4644/7643: Deep Learning\\
Spring 2022 \\
Problem Set 0}
\author{Instructor: Zsolt Kira \\
TAs: Bhavika Devnani, Jordan Rodrigues, Mandy Xie, \\Yanzhe Zhang, Amogh Dabholkar, Ahmed Shaikh, \\ Ting-Yu Lan, Anshul Ahluwalia, Aditya Singh \\
Discussions: \url{https://piazza.com/gatech/spring2022/cs46447643a}}
\date{Due: Thursday, Jan 13, 11:59pm ET}

\maketitle

\paragraph*{Instructions}
\begin{enumerate}
 \item We will be using Gradescope to collect your assignments.  Please read the following instructions for submitting to Gradescope carefully! Failure to follow these instructions may result in parts of your assignment not being graded. We will not entertain regrading requests for failure to follow instructions.
      \begin{itemize}
        \item
          For Section 1: Multiple Choice Questions, it is mandatory to use the {\LaTeX} template provided on the class webpage (\url{https://www.cc.gatech.edu/classes/AY2022/cs7643_spring/assets/ps0.zip}).
           For every question, there is only one correct answer. To mark the correct answer, change \codeword{\choice} to \codeword{\CorrectChoice}

           \item
                For Section 2: Proofs - This section has 7 total problems/sub-problems (Q8, Q9a - Q9c and Q10a - Q10c) and your answer to each sub-problem should start on a new page. Please be sure to mark the corresponding pages to the correct question numbers marked on the PS0 outline while submitting on Gradescope. 
           \item For Section 2, \LaTeX'd  solutions are strongly encouraged (solution template
           available at \\
           \url{https://www.cc.gatech.edu/classes/AY2022/cs7643_spring/assets/ps0.zip}),
           but scanned handwritten copies are acceptable. If you scan handwritten copies, please make sure to append them to the PDF generated by {\LaTeX} for Section 1 and please mark the pages correctly as mentioned above. 

      \end{itemize}

 \item Hard copies are \textbf{not} accepted.
\item We generally encourage you to collaborate with other students. You may talk to a friend,
discuss the questions and potential directions for solving them. However, you need to write
your own solutions and code separately, and \emph{not} as a group activity.
Please list the students you collaborated with. \\
\textbf{Exception: PS0 is meant to serve as a background preparation test. You must NOT collaborate on PS0.}

\end{enumerate}

%%%%%%%%%%%%%%%%%%%%%%%%%%%%%%%%%%%%%%%%%%%%%%%%%%%%%%%%%%%
% Body
%%%%%%%%%%%%%%%%%%%%%%%%%%%%%%%%%%%%%%%%%%%%%%%%%%%%%%%%%%%
\pagebreak

%%%%%%%%%%%%%%%%%%%%%%%%%%%%%%%%%%%%%%%%%%%%%%%%%%%%%%%%%%%
\section{Multiple Choice Questions}
%%%%%%%%%%%%%%%%%%%%%%%%%%%%%%%%%%%%%%%%%%%%%%%%%%%%%%%%%%%
\begin{enumerate}[start]

\item (1 point) Consider the tables below that display infection rates for a disease in two independent regions given vaccine status.
%
\\\\
\begin{tabular}{ |c|c|c|c| } 
	\hline
	Region & Pop. & Vaccination Rates & \% of Population Infected \\
	\hline
	Cityville & 874,961 & 77.0\% & 0.36\% \\ 
	\hline
	Townsland & 578,759 & 37.7\% &  1\%\\ 
	\hline
\end{tabular}

\vspace{0.5cm}

\begin{tabular}{ |c|c|c| } 
	\hline
	Region & \% of Infected people that are Vaccinated & \% of Infected people that are Unvaccinated\\
	\hline
	Cityville & 27.8\% & 72.2\%  \\ 
	\hline
	Townsland & 5.0\% & 95.0\%\\ 
	\hline
\end{tabular}

It appears that infected individuals in Cityville are much more likely to be vaccinated than in Townsland. Given these tables, would a vaccinated individual be less likely to be infected in Cityville or Townsland?


\begin{oneparcheckboxes}
\choice Cityville
\choice Townsland
\end{oneparcheckboxes}




\item (1 point)

Given a (possibly) biased coin with $P(Heads) = p$ and $P(Tails) = 1 - p$, first determine the method to generate a fair outcome (50:50) in the fewest amount of flips using this coin.
\\\\
What is the expected number of coin flips required (in terms of $p$) to produce a fair outcome using this method?

\vspace{0.5cm}

\begin{oneparcheckboxes}
\choice $\frac{1}{p(1-p)}$
\choice $\frac{1}{1 + p^2}$
\choice $\frac{2p}{1 - p}$
\choice A fair outcome cannot be generated with a biased coin
\end{oneparcheckboxes}

\vspace{0.5cm}

\item (1 point)
$X$ is a continuous random variable with probability density function:
%
\begin{align}
p(x) = \left\{ {\begin{array}{*{20}{c}}
{\begin{array}{*{20}{c}}
{2x^3/81}&{0 \le x \le 3}
\end{array}}\\
{\begin{array}{*{20}{c}}
{2(x-3)/8}&{3\le x \le 5}
\end{array}}
\end{array}} \right.
\end{align}
%
Which of the following statements are true about the equation for the corresponding cumulative density function (CDF) $C(x)$?\\
{[\emph{Hint:} Recall that CDF is defined as $C(x) = Pr(X \le x)$.]}\\
\begin{checkboxes}
	\choice $C(x) = x^4/162$  for   $0 \le x \le 3 $
	\choice $C(x) = x^2/8 -3x/4 + 13/8$ for  $3 \le x \le 5$
	\choice All of the above
	\choice None of the above
\end{checkboxes}

\item (2 point) A random variable $x$ in standard normal distribution 
has the following probability density:
%
\begin{align}
p(x) = \frac{1}{{\sqrt {2\pi } }}{e^{ - \frac{{{x^2}}}{2}}}
\end{align}
%
Evaluate the following integral:
%
\begin{align}
\int\limits_{ - \infty }^\infty  {p(x)(a{x^2} + b{x} - c)dx}
\end{align}

[\emph{Hint:} We are not sadistic (okay, we're a little sadistic, but not for this question). This is not a calculus question.]\\

\begin{oneparcheckboxes}
	\choice a + b + c
	\choice - c
	\choice a - c
	\choice b + c
\end{oneparcheckboxes}

\vspace{2cm}

	\item
	(2 points)
	Consider the following function of $\vec{x} = (x_1, x_2, x_3, x_4, x_5, x_6)$:
	\beqn
	f(\vec{x}) = \sigma\left( \log\left( 5 \left(
	\max\{x_1, x_2\} \cdot \frac{x_3}{x_4} - (x_5 + x_6) \right)
	\right) + \frac{1}{2} \right)
	\eeqn
	where $\sigma$ is the sigmoid function
	\beqn
	\sigma(x) = \frac{1}{1 + e^{-x}}
	\eeqn
	
	Compute the gradient $\nabla_{\vec{x}} f(\cdot)$ and evaluate it at at $\hat{\vec{x}} = (-1, 3, 4, 5, -5, 7)$.
	
\begin{oneparcheckboxes}
	\choice 
	$\begin{bmatrix}
		0 \\
		0.031 \\
		0.026 \\
		-0.013 \\
		-0.062 \\
		-0.062 \\
	\end{bmatrix}$
 \choice $\begin{bmatrix}
	0 \\
	0.157 \\
	0.131 \\
	-0.065 \\
	-0.314 \\
	-0.314 \\
	\end{bmatrix}$
	\choice $\begin{bmatrix}
	0 \\
	0.357 \\
	0.268 \\
	-0.214 \\
	-0.894 \\
	-0.894 \\
	\end{bmatrix}$
	\choice $\begin{bmatrix}
	0 \\
	0.357 \\
	0.268 \\
	-0.214 \\
	-0.447 \\
	-0.447 \\
	\end{bmatrix}$
\end{oneparcheckboxes}

\item (2 points) Which of the following functions are convex?
\begin{checkboxes}
\choice $\|\x\|_{\frac{1}{2}}$ 
\choice $\min_{i=1}^k \vec{a}_{i}^T \x$ for $\x \in \mathbb{R}^n$, and a finite set of arbitrary vectors: $\{\vec{a}_1, \dots, \vec{a}_k\}$
\choice $\log{(1+\exp(\wb^T \x))} $ for $\wb \in \mathbb{R}^d$
\choice All of the above
\end{checkboxes}


\item (2 points) Suppose you want to predict an unknown value $Y \in \RF$, but you are only given
a sequence of noisy observations $x_1, \dots, x_n$ of $Y$ with i.i.d.\ noise ($x_i = Y + \epsilon_i $). If we assume the noise is I.I.D.\ Gaussian ($\epsilon_i \sim N(0, \sigma^2)$), the maximum likelihood
estimate ($\hat y$) for $Y$ can be given by: 


\begin{checkboxes}
	\choice A: $ \hat y = \argmin_y \sum_{i=1}^{n} (y - x_i)^2 $ 
	\choice B: $ \hat y = \argmin_y \sum_{i=1}^{n} |y - x_i| $
	\choice C: $\hat y = \frac{1}{n} \sum_{i=1}^{n} x_i $
	\choice Both A \&  C
	\choice Both B \& C
	
\end{checkboxes}
\end{enumerate}
\pagebreak
%%%%%%%%%%%%%%%%%%%%%%%%%%%%%%%%%%%%%%%%%%%%%%%%%%%%%%%%%%%
\section{Proofs}
%%%%%%%%%%%%%%%%%%%%%%%%%%%%%%%%%%%%%%%%%%%%%%%%%%%%%%%%%%%

\begin{enumerate}[resume]
\item (3 points) Prove that
%
\begin{align}
\log_e x\leq x-1, \qquad \forall x>0
\end{align}
%
with equality if and only if $x=1$.

[\emph{Hint:} Consider differentiation of $\log(x)-(x-1)$ and think about concavity/convexity and second derivatives.]
\pagebreak
\item (6 points)
Consider two discrete probability distributions $p$ and $q$ over $k$ outcomes:
%
\begin{subequations}
\begin{align}
\sum_{i=1}^k p_i = \sum_{i=1}^k q_i=1 \\
p_i > 0, q_i > 0, \quad \forall i \in \{1,\ldots,k\}
\end{align}
\end{subequations}
%
The Kullback-Leibler (KL) divergence (also known as the \emph{relative entropy}) between these distributions is given by:
%
\begin{equation}
KL(p,q)=\sum_{i=1}^{k} p_i\log\left(\frac{p_i}{q_i}\right)
\end{equation}
%
It is common to refer to $KL(p,q)$ as a measure of distance (even though it is not a proper metric).
Many algorithms in machine learning are based on minimizing KL divergence between two
probability distributions.
In this question, we will show why this might be a sensible thing to do.\\

[\emph{Hint:} This question doesn't require you to know anything more than the definition of $KL(p,q)$ and the identity
in Q$7$]

\begin{enumerate}
\item Using the results from Q$7$, show that $KL(p,q)$ is always non-negative.
\pagebreak
\item When is $KL(p,q) = 0$?
\pagebreak
\item Provide a counterexample to show that the KL divergence is not a symmetric function of its arguments: $KL(p,q) \neq KL(q,p)$
\end{enumerate}

\pagebreak
\item
(6 points) In this question, we will get familiar with a fairly popular and useful function, called the log-sum-exp function. For $\vec{x} \in \mathbb{R}^n$, the log-sum-exp function is defined (quite literally) as:
\beqn
f(\vec{x}) = \log\bigg(\sum_{i=1}^n e^{x_i}\bigg)
\eeqn
\begin{enumerate}
\item Prove that $f(\vec{x})$ is differentiable everywhere in $\mathbb{R}^n$.

[\emph{Hint:} Multivariable functions are differentiable if the partial derivatives exist and are continuous.]
\pagebreak
\item Prove that $f(\vec{x})$ is convex on $\mathbb{R}^n$.

[\emph{Hint:} One approach is to use the second-order condition for convexity.]
\pagebreak
\item Show that $f(\vec{x})$ can be viewed as an approximation of the max function, bounded as follows:
\beqn
\max\{x_1, \dots, x_n\} \le f(\vec{x}) \le \max\{x_1, \dots, x_n\} + \log(n)
\eeqn
\pagebreak
\end{enumerate}
\end{enumerate}

%input{prob}
%\input{calculus}
%\input{softmax}

\end{document}
